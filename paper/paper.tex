\documentclass[12pt]{article}
\usepackage[utf8]{inputenc}
\usepackage{authblk}
\usepackage{graphicx}
\usepackage{setspace}
\doublespacing
\usepackage{geometry}
\geometry{legalpaper, margin=1in}

\title{idpflex: Analysis of Intrinsically Disordered Proteins by Comparing Simulations to Small Angle Scattering Experiments}
\author[1]{Jose M. Borreguero \thanks{Corresponding author. email: borreguerojm@ornl.gov}}
\author[1]{Fahima Islam}
\author[2]{Utsab R. Shrestha}
\author[2]{Loukas Petridis}
\affil[1]{Neutron Scattering Division, Oak Ridge National Laboratory, Oak Ridge TN, USA}
\affil[2]{Biosciences Division, Oak Ridge National Laboratory, Oak Ridge TN, USA}


\date{August 31 2018}

\begin{document}

\maketitle

\section{Summary}\label{summary}

It is estimated that about 30\% of the eucariotic proteome consists of intrinsically disordered proteins (IDP’s), yet their presence in public structural databases is severely underrepresented.
IDP’s adopt heterogeneous inter-converting conformations with similar probabilities, preventing resolution of structures with X-Ray diffraction techniques.
Small angle scattering (SAS) probes the average features of the conformational ensemble, which can
prove unsatisfactory when very different ensembles share nearly identical average features.
To address this shortcomings, atomistic molecular dynamics (MD) simulations combined with enhanced sampling methods such as the
Hamiltonian replica exchange method are specially suitable \cite{Affentranger06}, since they probe extensive regions of the IDP's free-energy phase space.
Hence, this type of simulations can produce physically meaningful conformations and offer a full-featured description of the conformational landscape when properly validated against available experimental SAS data.
The python package idpflex clusters the 3D conformationals resulting from an MD simulation into a
hierarchical tree, with protein substates taking the role of tree nodes.
Alternatively, the conformational ensemble of structures can be generated by other means than MD, such as torsional sampling of the protein backbone \cite{Curtis12}.
This flexibility is possible because idpflex is agnostic to the ensemble generation method.
In contrast to other methods \cite{Rozycki11}, idpflex does not initially discard any conformation by labelling it as incompatible with the available experimental data.
This data represents averages over the whole conformational ensemble, and using an average as the criterion by which to discard any specific conformation can lead to erroneous discarding decisions.
Clustering is performed according to structural similarity between conformations, defined by the root mean square deviation algorithm \cite{Kabsch76}.
Alternatively, idpflex can cluster conformations according to an Euclidean distance in the space spanned by a set of structural properties such as radius of gyration and end-to-end distance.
Calculation of SAS intensities \cite{Svergun95} for each substate allows quantitative comparison to SAS data, yielding the probability of the IDP to adopt one of the conformations of any specific substate.
Arranging tens of thousands of conformations into (typically) less than ten substates provides the researcher with a manageable number of macro-conformations from which to derive meaningful conclusions regarding the conformational freedom of the IDP.
In addition to clustering, idpflex can compute structural features for each substate such as radius of gyration, end-to-end distance, asphericity, solvent exposed surface area, contact maps, and secondary structure content.
All these properties require atomistic detail, thus idpflex is applicable to the study of IDP's but not to the study of quaternary protein arrangements, where clustering of coarse-grain simulations become much more appropriate \cite{Rozycki11}.
idpflex is extensible to include additional structural properties of interest.
In summary, idpflex integrates MD simulations with SAS experiments to obtain the conformational ensemble of IDP, a pertinent problem in structural biology.

The "notebooks" directory within the source contains two Jupyter notebooks that illustrate the use of idpflex when clustering an example MD trajectory.

\section{Notice of Copyright}\label{notice-of-copyright}

This manuscript has been authored by UT-Battelle, LLC under Contract No.
DE-AC05-00OR22725 with the U.S. Department of Energy. The United States
Government retains and the publisher, by accepting the article for
publication, acknowledges that the United States Government retains a
non-exclusive, paid-up, irrevocable, worldwide license to publish or
reproduce the published form of this manuscript, or allow others to do
so, for United States Government purposes. The Department of Energy will
provide public access to these results of federally sponsored research
in accordance with the DOE Public Access Plan
(http://energy.gov/downloads/doe-public-access-plan).

\section{Acknowledgements}\label{acknowledgements}
This work is sponsored by the Laboratory Directed Research and
Development Program of Oak Ridge National Laboratory, managed by
UT-Battelle LLC, for DOE. Part of this research is supported by the U.S.
Department of Energy, Office of Science, Office of Basic Energy
Sciences, User Facilities under contract number DE-AC05-00OR22725.

\bibliographystyle{unsrt}
\bibliography{paper}

\end{document}
